\section{Introdução}  
A crescente demanda por veículos híbridos e elétricos como alternativas mais sustentáveis aos veículos a combustão interna impulsionou o desenvolvimento de sistemas de gerenciamento de baterias de lítio mais eficientes e confiáveis. O gerenciamento adequado da temperatura das baterias de lítio é uma questão crítica para garantir seu desempenho e vida útil, bem como a segurança dos veículos elétricos.

A modelagem da temperatura em sistemas de gerenciamento de baterias de lítio é um desafio complexo devido à natureza não-linear e dinâmica do comportamento térmico das baterias, que é influenciado por diversos fatores, como a taxa de carga/descarga, temperatura ambiente, capacidade de carga, entre outros. Além disso, a incerteza e a variabilidade associadas a esses fatores tornam a modelagem térmica das baterias de lítio uma tarefa desafiadora.

Nesse contexto, a lógica fuzzy destaca-se como uma abordagem promissora para a modelagem de temperatura em sistemas de gerenciamento de baterias de lítio. A lógica fuzzy é uma técnica de modelagem matemática que permite lidar com a incerteza e a variabilidade dos dados, sendo capaz de representar e processar informações imprecisas e vagas. Dessa forma, a utilização de preditores baseados em lógica fuzzy pode ser uma abordagem eficiente para modelar o comportamento térmico das baterias de lítio em sistemas de gerenciamento, levando em consideração as incertezas e variabilidades inerentes a esse processo.

Diante desse contexto, este projeto de pesquisa propõe a modelagem de temperatura em sistemas de gerenciamento de baterias de lítio empregando preditores baseados em lógica fuzzy, com o objetivo de desenvolver um modelo robusto e preciso para prever a temperatura das baterias em veículos híbridos e elétricos. O uso de preditores baseados em lógica fuzzy pode proporcionar uma abordagem mais confiável para o gerenciamento térmico das baterias de lítio, contribuindo para o aumento do desempenho, segurança e vida útil desses sistemas em veículos híbridos e elétricos.

A pesquisa proposta busca contribuir para o avanço do conhecimento na área de engenharia elétrica, com ênfase em sistemas de gerenciamento de baterias de lítio e modelagem térmica. Além disso, o desenvolvimento de um modelo de predição de temperatura baseado em lógica fuzzy pode ter aplicações práticas na indústria automobilística, auxiliando no projeto e otimização de sistemas de gerenciamento de baterias de lítio para veículos híbridos e elétricos.