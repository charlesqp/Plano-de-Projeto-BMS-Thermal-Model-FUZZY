\section{Considerações Finais}
Uma das possíveis limitações deste estudo é a disponibilidade de dados para treinamento e validação dos modelos propostos. A obtenção de dados de temperatura de baterias de Lítio em tempo real pode ser desafiadora, visto que requer acesso a veículos elétricos ou híbridos, bem como permissões e autorizações para coleta de dados em ambientes reais de operação. A falta de dados de temperatura em diferentes condições de operação pode afetar a precisão e a confiabilidade dos modelos desenvolvidos.

Outra possível limitação é a complexidade dos modelos de lógica fuzzy e de aprendizado de máquina propostos. A implementação e validação desses modelos podem exigir habilidades técnicas e computacionais avançadas, bem como recursos computacionais adequados, como poder de processamento e capacidade de memória. A falta desses recursos pode representar um desafio na implementação prática dos modelos em ambientes reais de gerenciamento de baterias.

Além disso, é importante ressaltar que a aplicação dos modelos propostos em veículos elétricos e híbridos pode estar sujeita a regulamentações e normas específicas, que podem variar de acordo com a região geográfica e a legislação local. É fundamental considerar essas regulamentações e normas durante a implementação dos modelos propostos, a fim de garantir a conformidade com as exigências legais.

Outro desafio potencial é a natureza em constante evolução das tecnologias de baterias de Lítio e dos sistemas de gerenciamento de baterias. A rápida evolução dessas tecnologias pode afetar a relevância e a aplicabilidade dos modelos propostos em futuros cenários de veículos elétricos e híbridos. Portanto, é importante considerar a atualização e adaptação dos modelos propostos à medida que as tecnologias evoluem.

Apesar dessas possíveis limitações e desafios, a pesquisa proposta apresenta uma abordagem promissora para o gerenciamento de baterias de Lítio em veículos elétricos e híbridos, com o uso de preditores baseados em lógica fuzzy e modelos de aprendizado de máquina. Os resultados obtidos podem contribuir para o avanço do conhecimento na área, fornecendo insights valiosos para a otimização do desempenho térmico e a prolongação da vida útil das baterias. Futuras pesquisas podem abordar as limitações e desafios identificados, a fim de aprimorar ainda mais a aplicabilidade dos modelos propostos.