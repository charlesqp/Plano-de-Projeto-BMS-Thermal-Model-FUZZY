\section{Resultados Esperados}

O presente projeto de pesquisa tem como objetivo desenvolver um modelo de predição de temperatura em sistemas de gerenciamento de baterias de lítio empregando preditores baseados em lógica fuzzy para utilização em veículos híbridos e elétricos. Os resultados esperados deste projeto incluem:

\subsection{Desenvolvimento de um modelo de predição de temperatura baseado em lógica fuzzy}
 Espera-se que o projeto resulte no desenvolvimento de um modelo de predição de temperatura robusto e preciso, baseado em lógica fuzzy, que seja capaz de prever a temperatura das baterias de lítio em sistemas de gerenciamento. Esse modelo será desenvolvido com base em técnicas avançadas de lógica fuzzy, considerando as incertezas e variabilidades inerentes ao comportamento térmico das baterias, levando em consideração fatores como a taxa de carga/descarga, temperatura ambiente, capacidade de carga, entre outros.

\subsection{Avaliação e validação do modelo proposto}
Serão realizadas avaliações e validações detalhadas do modelo proposto, utilizando dados reais de temperatura de baterias de lítio em veículos híbridos e elétricos. Espera-se que o modelo desenvolvido seja capaz de apresentar resultados precisos e confiáveis na previsão da temperatura das baterias em diferentes condições de operação, contribuindo para a melhoria do gerenciamento térmico desses sistemas.

\subsection{Análise comparativa com outros modelos de predição}
Será realizada uma análise comparativa do modelo de predição de temperatura baseado em lógica fuzzy desenvolvido neste projeto com outros modelos de predição existentes na literatura e na indústria. Espera-se que o modelo proposto apresente vantagens em termos de precisão, robustez e confiabilidade na predição da temperatura das baterias de lítio em sistemas de gerenciamento.

\subsection{Contribuição para o conhecimento científico e aplicação prática}
Espera-se que os resultados obtidos com este projeto contribuam para o avanço do conhecimento na área de engenharia elétrica, especificamente na modelagem de temperatura em sistemas de gerenciamento de baterias de lítio. Além disso, espera-se que o modelo desenvolvido possa ter aplicação prática na indústria automobilística, auxiliando no projeto e otimização de sistemas de gerenciamento de baterias de lítio para veículos híbridos e elétricos, contribuindo assim para o desenvolvimento de tecnologias mais eficientes e seguras de mobilidade elétrica.
