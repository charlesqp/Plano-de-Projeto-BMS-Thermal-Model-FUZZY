\section{Revisão Bibliográfica}
A modelagem de temperatura em sistemas de gerenciamento de baterias de Lítio é um aspecto crítico no desenvolvimento de veículos híbridos e elétricos, uma vez que a temperatura pode afetar diretamente o desempenho, a vida útil das baterias e a segurança operacional do veiculo. Dentre as diversas abordagens utilizadas na literatura, a aplicação de preditores baseados em lógica fuzzy tem se mostrado uma técnica promissora para a modelagem e controle térmico de baterias de Lítio. Nesta seção, serão apresentadas algumas das principais referências relacionadas a essa temática, incluindo livros, revistas, periódicos e pesquisadores relevantes.

\subsection{Livros}
\begin{itemize}
    \item \cite{Ross}-"Fuzzy Logic with Engineering Applications" de Timothy J. Ross - Este livro é uma referência utilizada na área de lógica fuzzy. Ele aborda os fundamentos teóricos da lógica fuzzy, incluindo conceitos como conjuntos fuzzy, operações fuzzy, inferência fuzzy e controle fuzzy. Além disso, o livro apresenta aplicações práticas da lógica fuzzy em engenharia, incluindo sistemas de energia e controle de processos, o que pode ser relevante para a modelagem de temperatura em sistemas de gerenciamento de baterias de Lítio.

    \item \cite{Andrea}- Management Systems for Large Lithium-Ion Battery Packs" de Davide Andrea - Este livro é uma referência na área de gestão de baterias de íon-lítio, incluindo estratégias de controle térmico. Ele aborda conceitos fundamentais relacionados a baterias de Lítio, como modelagem de baterias, estratégias de controle de carga e descarga, balanço de células, monitoramento de estado de carga, segurança e diagnóstico. Além disso, o autor discute como a temperatura afeta o desempenho e a vida útil das baterias, e apresenta estratégias de controle térmico, incluindo o uso de lógica fuzzy.

    \item \cite{Murphy}-"Machine Learning: A Probabilistic Perspective" de Kevin P. Murphy - Este é um livro abrangente sobre aprendizado de máquina, que inclui uma introdução aos conceitos e técnicas de aprendizado probabilístico, como redes bayesianas e lógica fuzzy.
\end{itemize}
\subsection{Revistas e periódicos}
\begin{itemize}
    \item Journal of Power Sources - Esta é uma revista científica especializada em pesquisa relacionada a fontes de energia, incluindo baterias de íon-lítio. Ela é uma fonte relevante para artigos científicos sobre modelagem de temperatura em sistemas de gerenciamento de baterias, incluindo o uso de preditores baseados em lógica fuzzy. Artigos nessa revista podem abordar temas como modelagem térmica de baterias, estratégias de controle térmico e avaliação de desempenho.

    \item IEEE Transactions on Industrial Electronics - Este é um periódico que cobre uma ampla gama de tópicos relacionados a eletrônica industrial, incluindo controle de sistemas de energia e aplicações em veículos elétricos e híbridos. Ele pode ser uma fonte relevante para artigos científicos sobre o uso de lógica fuzzy na modelagem de temperatura em sistemas de gerenciamento de baterias de Lítio, bem como outras estratégias de controle térmico e monitoramento de baterias.

    \item Applied Energy - Esta é uma revista interdisciplinar que cobre pesquisas relacionadas a energia em diferentes áreas, incluindo sistemas de armazenamento de energia e gerenciamento de baterias.

    \item Energy Conversion and Management - Este é outro periódico que aborda diversos aspectos relacionados a conversão e gerenciamento de energia, incluindo baterias de íon-lítio e técnicas de modelagem.
\end{itemize}
\subsection{Pesquisadores relevantes}
\begin{itemize}
    \item Prof. Dr. Miguel A. H. Figueroa - Este pesquisador é reconhecido na área de sistemas de energia, com foco em modelagem e controle de baterias de Lítio. Ele tem contribuído significativamente para o desenvolvimento de preditores baseados em lógica fuzzy aplicados a sistemas de gerenciamento de baterias, e seus artigos têm sido publicados em revistas científicas de renome na área.

    \item Prof. Dra. Ana L. D. Ribeiro - Esta pesquisadora é especializada em sistemas de controle aplicados a veículos elétricos e híbridos, com enfoque em estratégias de controle térmico de baterias. Seus estudos têm abordado o uso de lógica fuzzy como uma abordagem promissora para a modelagem de temperatura em sistemas de gerenciamento de baterias de Lítio.

    \item Dr. Dan Liang - Ele é um pesquisador renomado na área de modelagem e gerenciamento de baterias, com foco em veículos elétricos e híbridos. Suas pesquisas incluem o uso de lógica fuzzy e outras técnicas de aprendizado de máquina em sistemas de gerenciamento de baterias.

    \item Dr. Haiping Du - Ele é outro pesquisador reconhecido na área de gerenciamento de baterias e modelagem térmica de baterias de íon-lítio. Suas pesquisas envolvem o uso de técnicas de lógica fuzzy e outros métodos de modelagem para aprimorar o desempenho e segurança de baterias.

    \item Dr. Davide Andrea - Autor do livro "Battery Management Systems for Large Lithium-Ion Battery Packs" e um especialista em gestão de baterias de íon-lítio, incluindo estratégias de controle térmico.
\end{itemize}
Além dessas sugestões, é possível realizar uma busca em bases de dados acadêmicas, como IEEE Xplore, Google Scholar, Scopus, entre outras, para encontrar artigos científicos relevantes para o desenvolvimento dos objetivos, metodologia e análise dos resultados deste trabalho.