\documentclass[11pt,a4paper]{article}

\usepackage[brazil]{babel}
\usepackage[utf8]{inputenc}
\usepackage{ae}
\usepackage{indentfirst}
\usepackage[margin=2cm]{geometry}
\usepackage{amsthm,amssymb,amsfonts,amsmath}
\usepackage{graphicx}
\usepackage{psfrag}
\usepackage{booktabs}
%\usepackage[numbers]{natbib}
\usepackage{enumitem}
\usepackage{colortbl,booktabs}
\usepackage{enumerate}
\usepackage{enumitem}
\usepackage{pgfgantt}
\usepackage[
backend=biber,
style=alphabetic,
sorting=ynt
]{biblatex}
\usepackage{csquotes}
\addbibresource{manualbib.bib} %Import the bibliography file
\addbibresource{MDLreferences.bib} %Import the bibliography file

\newlist{inparaenum}{enumerate}{2}% allow two levels of nesting in an enumerate-like environment
\setlist[inparaenum]{nosep}% compact spacing for all nesting levels
\setlist[inparaenum,1]{label=\bfseries\arabic*.}% labels for top level
\setlist[inparaenum,2]{label=\arabic{inparaenumi}.\emph{\alph*})}% labels for second level

\NewDocumentCommand{\tituloProjeto}{}{Projeto de observadores baseados em lógica fuzzy para estimação de temperatura em sistemas de gerenciamento de baterias de Lítio para utilização em veículos híbridos e elétricos}
\NewDocumentCommand{\nomeAluno}{}{Charles Quirino Pimenta}
\NewDocumentCommand{\nomeOrientador}{}{- A DEFINIR -}

\begin{document}

\input{00_capa}
\section{Introdução}  
A crescente demanda por veículos híbridos e elétricos como alternativas mais sustentáveis aos veículos a combustão interna impulsionou o desenvolvimento de sistemas de gerenciamento de baterias de lítio mais eficientes e confiáveis. O gerenciamento adequado da temperatura das baterias de lítio é uma questão crítica para garantir seu desempenho e vida útil, bem como a segurança dos veículos elétricos.

A modelagem da temperatura em sistemas de gerenciamento de baterias de lítio é um desafio complexo devido à natureza não-linear e dinâmica do comportamento térmico das baterias, que é influenciado por diversos fatores, como a taxa de carga/descarga, temperatura ambiente, capacidade de carga, entre outros. Além disso, a incerteza e a variabilidade associadas a esses fatores tornam a modelagem térmica das baterias de lítio uma tarefa desafiadora.

Nesse contexto, a lógica fuzzy destaca-se como uma abordagem promissora para a modelagem de temperatura em sistemas de gerenciamento de baterias de lítio. A lógica fuzzy é uma técnica de modelagem matemática que permite lidar com a incerteza e a variabilidade dos dados, sendo capaz de representar e processar informações imprecisas e vagas. Dessa forma, a utilização de preditores baseados em lógica fuzzy pode ser uma abordagem eficiente para modelar o comportamento térmico das baterias de lítio em sistemas de gerenciamento, levando em consideração as incertezas e variabilidades inerentes a esse processo.

Diante desse contexto, este projeto de pesquisa propõe a modelagem de temperatura em sistemas de gerenciamento de baterias de lítio empregando preditores baseados em lógica fuzzy, com o objetivo de desenvolver um modelo robusto e preciso para prever a temperatura das baterias em veículos híbridos e elétricos. O uso de preditores baseados em lógica fuzzy pode proporcionar uma abordagem mais confiável para o gerenciamento térmico das baterias de lítio, contribuindo para o aumento do desempenho, segurança e vida útil desses sistemas em veículos híbridos e elétricos.

A pesquisa proposta busca contribuir para o avanço do conhecimento na área de engenharia elétrica, com ênfase em sistemas de gerenciamento de baterias de lítio e modelagem térmica. Além disso, o desenvolvimento de um modelo de predição de temperatura baseado em lógica fuzzy pode ter aplicações práticas na indústria automobilística, auxiliando no projeto e otimização de sistemas de gerenciamento de baterias de lítio para veículos híbridos e elétricos.
\section{Formulação do Problema}

A bateria de íons de lítio é uma das fontes de energia mais usadas em VEs devido ao seu ótimo desempenho. No entanto, é muito perigoso quando a bateria de íons de lítio está funcionando em um ambiente de alta temperatura.

Enquanto isso, a alta temperatura também reduzirá irreversivelmente a capacidade da bateria. Portanto, é especialmente importante controlar a temperatura da bateria dentro da faixa de operação segura, e o sistema de gerenciamento térmico da bateria é necessário para a bateria em veículos elétricos.

No sistema de gerenciamento térmico da bateria, um modelo de alta precisão é fundamental para controlar a temperatura da bateria. O modelo térmico é usado para calcular a taxa de geração de calor e a taxa de dissipação de calor para o pacote de bateria. Existem muitas reações eletroquímicas no processo de carga e descarga da bateria. O modelo eletroquímico descreve essas reações por meio de equações diferenciais parciais e pode atingir alta precisão. No entanto, a complexidade do modelo eletroquímico limita sua aplicação em tempo real. Devido a este nível de complexidade de implementação e processamento, se faz necessária a busca por uma abordagem que seja computacionalmente mais econômica, porem sem perda significativa de precisão.

Desta forma, o objetivo deste projeto é modelar a temperatura em sistemas de gerenciamento de baterias de Lítio empregando preditores baseados em lógica fuzzy para utilização em veículos híbridos e elétricos.

\subsection{Tópicos de investigação}

Especificamente, o projeto busca responder às seguintes questões de pesquisa:
 
\begin{itemize}
\item  Quais são os principais desafios relacionados à modelagem de temperatura em sistemas de gerenciamento de baterias de Lítio para veículos híbridos e elétricos?
\item  Como é possível utilizar preditores baseados em lógica fuzzy para otimizar o gerenciamento térmico de baterias de Lítio em veículos híbridos e elétricos?
\item  Quais são as técnicas e abordagens de lógica fuzzy mais adequadas para modelar a temperatura em sistemas de gerenciamento de baterias de Lítio?
\item  Como validar a eficácia dos preditores baseados em lógica fuzzy propostos na modelagem de temperatura em sistemas de gerenciamento de baterias de Lítio?
\end{itemize}

\section{Revisão Bibliográfica}
A modelagem de temperatura em sistemas de gerenciamento de baterias de Lítio é um aspecto crítico no desenvolvimento de veículos híbridos e elétricos, uma vez que a temperatura pode afetar diretamente o desempenho, a vida útil das baterias e a segurança operacional do veiculo. Dentre as diversas abordagens utilizadas na literatura, a aplicação de preditores baseados em lógica fuzzy tem se mostrado uma técnica promissora para a modelagem e controle térmico de baterias de Lítio. Nesta seção, serão apresentadas algumas das principais referências relacionadas a essa temática, incluindo livros, revistas, periódicos e pesquisadores relevantes.

\subsection{Livros}
\begin{itemize}
    \item \cite{Ross}-"Fuzzy Logic with Engineering Applications" de Timothy J. Ross - Este livro é uma referência utilizada na área de lógica fuzzy. Ele aborda os fundamentos teóricos da lógica fuzzy, incluindo conceitos como conjuntos fuzzy, operações fuzzy, inferência fuzzy e controle fuzzy. Além disso, o livro apresenta aplicações práticas da lógica fuzzy em engenharia, incluindo sistemas de energia e controle de processos, o que pode ser relevante para a modelagem de temperatura em sistemas de gerenciamento de baterias de Lítio.

    \item \cite{Andrea}- Management Systems for Large Lithium-Ion Battery Packs" de Davide Andrea - Este livro é uma referência na área de gestão de baterias de íon-lítio, incluindo estratégias de controle térmico. Ele aborda conceitos fundamentais relacionados a baterias de Lítio, como modelagem de baterias, estratégias de controle de carga e descarga, balanço de células, monitoramento de estado de carga, segurança e diagnóstico. Além disso, o autor discute como a temperatura afeta o desempenho e a vida útil das baterias, e apresenta estratégias de controle térmico, incluindo o uso de lógica fuzzy.

    \item \cite{Murphy}-"Machine Learning: A Probabilistic Perspective" de Kevin P. Murphy - Este é um livro abrangente sobre aprendizado de máquina, que inclui uma introdução aos conceitos e técnicas de aprendizado probabilístico, como redes bayesianas e lógica fuzzy.
\end{itemize}
\subsection{Revistas e periódicos}
\begin{itemize}
    \item Journal of Power Sources - Esta é uma revista científica especializada em pesquisa relacionada a fontes de energia, incluindo baterias de íon-lítio. Ela é uma fonte relevante para artigos científicos sobre modelagem de temperatura em sistemas de gerenciamento de baterias, incluindo o uso de preditores baseados em lógica fuzzy. Artigos nessa revista podem abordar temas como modelagem térmica de baterias, estratégias de controle térmico e avaliação de desempenho.

    \item IEEE Transactions on Industrial Electronics - Este é um periódico que cobre uma ampla gama de tópicos relacionados a eletrônica industrial, incluindo controle de sistemas de energia e aplicações em veículos elétricos e híbridos. Ele pode ser uma fonte relevante para artigos científicos sobre o uso de lógica fuzzy na modelagem de temperatura em sistemas de gerenciamento de baterias de Lítio, bem como outras estratégias de controle térmico e monitoramento de baterias.

    \item Applied Energy - Esta é uma revista interdisciplinar que cobre pesquisas relacionadas a energia em diferentes áreas, incluindo sistemas de armazenamento de energia e gerenciamento de baterias.

    \item Energy Conversion and Management - Este é outro periódico que aborda diversos aspectos relacionados a conversão e gerenciamento de energia, incluindo baterias de íon-lítio e técnicas de modelagem.
\end{itemize}
\subsection{Pesquisadores relevantes}
\begin{itemize}
    \item Prof. Dr. Miguel A. H. Figueroa - Este pesquisador é reconhecido na área de sistemas de energia, com foco em modelagem e controle de baterias de Lítio. Ele tem contribuído significativamente para o desenvolvimento de preditores baseados em lógica fuzzy aplicados a sistemas de gerenciamento de baterias, e seus artigos têm sido publicados em revistas científicas de renome na área.

    \item Prof. Dra. Ana L. D. Ribeiro - Esta pesquisadora é especializada em sistemas de controle aplicados a veículos elétricos e híbridos, com enfoque em estratégias de controle térmico de baterias. Seus estudos têm abordado o uso de lógica fuzzy como uma abordagem promissora para a modelagem de temperatura em sistemas de gerenciamento de baterias de Lítio.

    \item Dr. Dan Liang - Ele é um pesquisador renomado na área de modelagem e gerenciamento de baterias, com foco em veículos elétricos e híbridos. Suas pesquisas incluem o uso de lógica fuzzy e outras técnicas de aprendizado de máquina em sistemas de gerenciamento de baterias.

    \item Dr. Haiping Du - Ele é outro pesquisador reconhecido na área de gerenciamento de baterias e modelagem térmica de baterias de íon-lítio. Suas pesquisas envolvem o uso de técnicas de lógica fuzzy e outros métodos de modelagem para aprimorar o desempenho e segurança de baterias.

    \item Dr. Davide Andrea - Autor do livro "Battery Management Systems for Large Lithium-Ion Battery Packs" e um especialista em gestão de baterias de íon-lítio, incluindo estratégias de controle térmico.
\end{itemize}
Além dessas sugestões, é possível realizar uma busca em bases de dados acadêmicas, como IEEE Xplore, Google Scholar, Scopus, entre outras, para encontrar artigos científicos relevantes para o desenvolvimento dos objetivos, metodologia e análise dos resultados deste trabalho.    
\section{Metodologia}
\subsection{Revisão Bibliográfica}
Nesta etapa, será realizada uma revisão bibliográfica abrangente sobre os conceitos fundamentais relacionados à modelagem de temperatura em sistemas de gerenciamento de baterias de Lítio, preditores baseados em lógica fuzzy, e utilização de baterias de Lítio em veículos híbridos e elétricos. Serão revisados os principais trabalhos científicos, artigos, livros e outras fontes relevantes relacionadas ao tema, com o objetivo de obter uma compreensão aprofundada das técnicas e abordagens existentes, identificar os desafios atuais e as lacunas de pesquisa no campo.

\subsection{Seleção e Preparação de Dados}
Nesta etapa, serão selecionados os dados necessários para o projeto de pesquisa. Será definido o conjunto de dados que será utilizado para treinamento, validação e teste dos preditores baseados em lógica fuzzy propostos. Os dados podem incluir informações sobre temperatura, estado de carga (SOC), corrente, tensão e outras variáveis relevantes para a modelagem de temperatura em sistemas de gerenciamento de baterias de Lítio em veículos híbridos e elétricos. Será realizada a preparação dos dados, incluindo a limpeza, normalização e transformação dos dados, conforme necessário.

\subsection{Desenvolvimento de Preditores Baseados em Lógica Fuzzy}
Nesta etapa, serão desenvolvidos os preditores baseados em lógica fuzzy para modelar a temperatura em sistemas de gerenciamento de baterias de Lítio. Serão exploradas diferentes técnicas e abordagens de lógica fuzzy, como sistemas de inferência fuzzy, conjuntos fuzzy, regras fuzzy e métodos de defuzzificação, para projetar e implementar os preditores fuzzy. Será realizado o treinamento dos preditores utilizando o conjunto de dados selecionado na etapa anterior, ajustando os parâmetros e otimizando o desempenho dos preditores.

\subsection{Avaliação e Validação dos Preditores Propostos}
Nesta etapa, serão realizadas avaliações e validações dos preditores baseados em lógica fuzzy propostos. Será avaliado o desempenho dos preditores em relação a métricas de desempenho relevantes, como erro médio, erro quadrático médio, coeficiente de determinação, entre outros. Serão realizados experimentos e simulações para comparar os preditores propostos com outras técnicas de modelagem de temperatura em sistemas de gerenciamento de baterias de Lítio, como modelos matemáticos tradicionais ou outros modelos de aprendizado de máquina. Serão feitas análises estatísticas para avaliar a significância dos resultados obtidos. A validação pode ser feita por meio de comparação dos resultados obtidos pelos modelos propostos com dados de referência, obtidos de medições reais em sistemas de gerenciamento de baterias de Lítio. Além disso, podem ser utilizadas técnicas de validação cruzada, onde os dados são divididos em conjuntos de treinamento e teste, para avaliar o desempenho dos modelos em dados não utilizados no treinamento.

\subsection{Análise e Interpretação dos Resultados}
Nesta etapa, os resultados obtidos serão analisados e interpretados. Serão discutidos os principais achados e conclusões do estudo em relação aos objetivos de pesquisa propostos. Serão identificadas as contribuições e limitações dos preditores baseados em lógica fuzzy propostos, e serão fornecidas recomendações para possíveis melhorias e futuras pesquisas. A análise dos resultados será embasada em análises estatísticas, gráficos, tabelas e outros métodos adequados para a interpretação dos resultados obtidos.

\subsection{Discussão dos Resultados à Luz da Revisão Bibliográfica}
Nesta etapa, os resultados obtidos serão discutidos em relação à revisão bibliográfica realizada na etapa de revisão bibliográfica. Será feita uma análise crítica dos resultados em relação aos trabalhos científicos e fontes relevantes revisadas, destacando as semelhanças, diferenças e inovações dos preditores baseados em lógica fuzzy propostos em comparação com as abordagens existentes na literatura. Será fornecido embasamento teórico para apoiar os resultados obtidos e as conclusões do estudo.

%\subsection{Considerações Éticas}
%Nesta etapa, serão consideradas as questões éticas relacionadas à pesquisa, especialmente no que diz respeito à utilização dos dados, privacidade, consentimento informado, e outras questões éticas relevantes para a pesquisa em engenharia elétrica. Será garantido o cumprimento de todas as normas éticas e regulatórias aplicáveis à pesquisa, e serão tomadas medidas adequadas para proteger a privacidade e os direitos dos participantes envolvidos na coleta e uso dos dados.

%\subsection{Cronograma}
%Será elaborado um cronograma detalhado para o desenvolvimento do projeto de pesquisa, com a identificação das atividades a serem realizadas, os prazos de execução e a sequência lógica das etapas do projeto. O cronograma servirá como um guia para o planejamento e acompanhamento do projeto, auxiliando na organização e gestão do tempo.

%\subsection{Recursos}
%Será identificado e listado os recursos necessários para a realização do projeto de pesquisa, como equipamentos, softwares, materiais, acesso a bases de dados, entre outros. Será indicada a disponibilidade e acessibilidade dos recursos, bem como os meios para obtê-los, caso necessário.  
\section{Resultados Esperados}

O presente projeto de pesquisa tem como objetivo desenvolver um modelo de predição de temperatura em sistemas de gerenciamento de baterias de lítio empregando preditores baseados em lógica fuzzy para utilização em veículos híbridos e elétricos. Os resultados esperados deste projeto incluem:

\subsection{Desenvolvimento de um modelo de predição de temperatura baseado em lógica fuzzy}
 Espera-se que o projeto resulte no desenvolvimento de um modelo de predição de temperatura robusto e preciso, baseado em lógica fuzzy, que seja capaz de prever a temperatura das baterias de lítio em sistemas de gerenciamento. Esse modelo será desenvolvido com base em técnicas avançadas de lógica fuzzy, considerando as incertezas e variabilidades inerentes ao comportamento térmico das baterias, levando em consideração fatores como a taxa de carga/descarga, temperatura ambiente, capacidade de carga, entre outros.

\subsection{Avaliação e validação do modelo proposto}
Serão realizadas avaliações e validações detalhadas do modelo proposto, utilizando dados reais de temperatura de baterias de lítio em veículos híbridos e elétricos. Espera-se que o modelo desenvolvido seja capaz de apresentar resultados precisos e confiáveis na previsão da temperatura das baterias em diferentes condições de operação, contribuindo para a melhoria do gerenciamento térmico desses sistemas.

\subsection{Análise comparativa com outros modelos de predição}
Será realizada uma análise comparativa do modelo de predição de temperatura baseado em lógica fuzzy desenvolvido neste projeto com outros modelos de predição existentes na literatura e na indústria. Espera-se que o modelo proposto apresente vantagens em termos de precisão, robustez e confiabilidade na predição da temperatura das baterias de lítio em sistemas de gerenciamento.

\subsection{Contribuição para o conhecimento científico e aplicação prática}
Espera-se que os resultados obtidos com este projeto contribuam para o avanço do conhecimento na área de engenharia elétrica, especificamente na modelagem de temperatura em sistemas de gerenciamento de baterias de lítio. Além disso, espera-se que o modelo desenvolvido possa ter aplicação prática na indústria automobilística, auxiliando no projeto e otimização de sistemas de gerenciamento de baterias de lítio para veículos híbridos e elétricos, contribuindo assim para o desenvolvimento de tecnologias mais eficientes e seguras de mobilidade elétrica.

\section{Plano de trabalho e cronograma}

Este plano de mestrado está previsto para doze meses de duração,
compreendendo as seguintes etapas:
 
\renewcommand{\labelenumi}{\alph{enumi})}
\begin{enumerate} 
    \item Obtenção dos créditos necessários
    \item Revisão bibliográfica sobre sistemas de gerenciamento de baterias de Lítio, lógica fuzzy e modelos de predição de temperatura
    \item Definição dos objetivos do trabalho e metodologia de pesquisa
    \item Coleta de dados de temperatura de baterias de Lítio em veículos híbridos e elétricos
    \item Desenvolvimento de preditores baseados em lógica fuzzy para modelagem de temperatura
    \item Implementação dos modelos de aprendizado de máquina
    \item Treinamento e ajuste dos modelos com os dados coletados
    \item Análise dos resultados obtidos e comparação com outros métodos
    \item Redação da dissertação
    \item Revisões e correções na dissertação
    \item Apresentação e defesa do trabalho de pesquisa
\end{enumerate}
 
 O cronograma apresenta, aproximadamente, a distribuição das etapas descritas ao longo dos meses.
\definecolor{RoyalBlue}{rgb}{0.25, 0.41, 0.88}
\definecolor{black}{rgb}{0.0, 0.0, 0.0}
\definecolor{OliveGreen}{rgb}{0,0.6,0}
\definecolor{Maroon}{rgb}{0.5, 0.0, 0.0}

\def\mystartdate{2022-08}%starting date of the calendar
\def\myenddate{2024-08}%ending date of the calendar

\begin{ganttchart}[
y unit title=0.4cm,
y unit chart=0.5cm,
vgrid,
expand chart=\textwidth,
time slot format=isodate-yearmonth,
time slot unit=month,
progress=today,
title/.append style={draw=none, fill=RoyalBlue!50!black},
title label font=\sffamily\bfseries\color{white},
title label node/.append style={below=-1.6ex},
title left shift=.05,
title right shift=-.05,
title height=1,
bar/.append style={draw=none, fill=OliveGreen!75},
bar height=.6,
bar label font=\normalsize\color{black!50},
group right shift=0,
group top shift=.6,
group height=.3,
group peaks height=.2,
bar incomplete/.append style={fill=Maroon},
today={2023-04}
]{\mystartdate}{\myenddate}
    \gantttitlecalendar{year, month} \\
    \ganttbar{a)}{2022-08-01}{2023-06-31} \\ %  Obtenção dos créditos necessários
    \ganttbar{b)}{2023-05}{2023-05} \\ % {1} Revisão bibliográfica
    \ganttbar{c)}{2023-06}{2023-06} \\ % {1} Definição dos objetivos e metodologia de pesquisa
    \ganttbar{d)}{2023-07}{2023-07} \\ % {3} Coleta de dados de temperatura
    \ganttbar{e)}{2023-08}{2023-09} \\ % {2} Desenvolvimento de preditores baseados em lógica fuzzy 
    \ganttbar{f)}{2023-10}{2023-10} \\ % {1} Implementação dos modelos de aprendizado de máquina
    \ganttbar{g)}{2023-11}{2023-12} \\ % {2} Treinamento e ajuste dos modelos com os dados coletados
    \ganttbar{h)}{2024-01}{2024-01} \\ % {1} Análise dos resultados obtidos e comparação com outros métodos
    \ganttbar{i)}{2024-02}{2024-05} \\ % {6} Redação da dissertação/tese
    \ganttbar{i)}{2024-06}{2024-06} \\ % {6} Revisões e correções na dissertação/tese
    \ganttbar{i)}{2024-07}{2024-07} \\ % {6} Apresentação e defesa do trabalho de pesquisa
\end{ganttchart}
\section{Orçamento}

Estimativa dos recursos financeiros necessários para a realização do projeto de pesquisa, incluindo custos com materiais, equipamentos, deslocamentos, e outros gastos pertinentes.
  
\begin{inparaenum}
    \item Materiais
    \begin{inparaenum}
        \item Livros, artigos científicos e outras publicações relacionadas à pesquisa: R\$ 500,00
        \item Materiais de consumo (papelaria, impressão, mídias, etc.): R\$ 300,00
        \item Total estimado de custos com materiais: R\$ 800,00
    \end{inparaenum}
        
    \item Equipamentos 
    \begin{inparaenum}
        \item Não é estimado custos com equipamentos nesse projeto, uma vez que se trata de uma pesquisa teórica e não requer aquisição de equipamentos específicos.
        \item Total estimado de custos com equipamentos: R\$ 800,00
    \end{inparaenum}
    
    \item Deslocamentos
    \begin{inparaenum}
        \item Transporte para coleta de dados ou visitas a laboratórios: R\$ 1.000,00 (considerando deslocamentos dentro do estado de Minas Gerais)
        \item Total estimado de custos com deslocamentos: R\$ 1.000,00
    \end{inparaenum}
    
    \item Outros gastos
    \begin{inparaenum}
        \item Taxas de inscrição em eventos científicos: R\$ 1.000,00
        \item Total estimado de outros gastos: R\$ 1.000,00
    \end{inparaenum}
\end{inparaenum}
 
\paragraph{Total}
Total estimado do orçamento do projeto: R\$ 2.800,00

\section{Considerações Finais}
Uma das possíveis limitações deste estudo é a disponibilidade de dados para treinamento e validação dos modelos propostos. A obtenção de dados de temperatura de baterias de Lítio em tempo real pode ser desafiadora, visto que requer acesso a veículos elétricos ou híbridos, bem como permissões e autorizações para coleta de dados em ambientes reais de operação. A falta de dados de temperatura em diferentes condições de operação pode afetar a precisão e a confiabilidade dos modelos desenvolvidos.

Outra possível limitação é a complexidade dos modelos de lógica fuzzy e de aprendizado de máquina propostos. A implementação e validação desses modelos podem exigir habilidades técnicas e computacionais avançadas, bem como recursos computacionais adequados, como poder de processamento e capacidade de memória. A falta desses recursos pode representar um desafio na implementação prática dos modelos em ambientes reais de gerenciamento de baterias.

Além disso, é importante ressaltar que a aplicação dos modelos propostos em veículos elétricos e híbridos pode estar sujeita a regulamentações e normas específicas, que podem variar de acordo com a região geográfica e a legislação local. É fundamental considerar essas regulamentações e normas durante a implementação dos modelos propostos, a fim de garantir a conformidade com as exigências legais.

Outro desafio potencial é a natureza em constante evolução das tecnologias de baterias de Lítio e dos sistemas de gerenciamento de baterias. A rápida evolução dessas tecnologias pode afetar a relevância e a aplicabilidade dos modelos propostos em futuros cenários de veículos elétricos e híbridos. Portanto, é importante considerar a atualização e adaptação dos modelos propostos à medida que as tecnologias evoluem.

Apesar dessas possíveis limitações e desafios, a pesquisa proposta apresenta uma abordagem promissora para o gerenciamento de baterias de Lítio em veículos elétricos e híbridos, com o uso de preditores baseados em lógica fuzzy e modelos de aprendizado de máquina. Os resultados obtidos podem contribuir para o avanço do conhecimento na área, fornecendo insights valiosos para a otimização do desempenho térmico e a prolongação da vida útil das baterias. Futuras pesquisas podem abordar as limitações e desafios identificados, a fim de aprimorar ainda mais a aplicabilidade dos modelos propostos.

%\bibliographystyle{plainnat}
\printbibliography %Prints bibliography
\end{document}