\section{Formulação do Problema}

A bateria de íons de lítio é uma das fontes de energia mais usadas em VEs devido ao seu ótimo desempenho. No entanto, é muito perigoso quando a bateria de íons de lítio está funcionando em um ambiente de alta temperatura.

Enquanto isso, a alta temperatura também reduzirá irreversivelmente a capacidade da bateria. Portanto, é especialmente importante controlar a temperatura da bateria dentro da faixa de operação segura, e o sistema de gerenciamento térmico da bateria é necessário para a bateria em veículos elétricos.

No sistema de gerenciamento térmico da bateria, um modelo de alta precisão é fundamental para controlar a temperatura da bateria. O modelo térmico é usado para calcular a taxa de geração de calor e a taxa de dissipação de calor para o pacote de bateria. Existem muitas reações eletroquímicas no processo de carga e descarga da bateria. O modelo eletroquímico descreve essas reações por meio de equações diferenciais parciais e pode atingir alta precisão. No entanto, a complexidade do modelo eletroquímico limita sua aplicação em tempo real. Devido a este nível de complexidade de implementação e processamento, se faz necessária a busca por uma abordagem que seja computacionalmente mais econômica, porem sem perda significativa de precisão.

Desta forma, o objetivo deste projeto é modelar a temperatura em sistemas de gerenciamento de baterias de Lítio empregando preditores baseados em lógica fuzzy para utilização em veículos híbridos e elétricos.

\subsection{Tópicos de investigação}

Especificamente, o projeto busca responder às seguintes questões de pesquisa:
 
\begin{itemize}
\item  Quais são os principais desafios relacionados à modelagem de temperatura em sistemas de gerenciamento de baterias de Lítio para veículos híbridos e elétricos?
\item  Como é possível utilizar preditores baseados em lógica fuzzy para otimizar o gerenciamento térmico de baterias de Lítio em veículos híbridos e elétricos?
\item  Quais são as técnicas e abordagens de lógica fuzzy mais adequadas para modelar a temperatura em sistemas de gerenciamento de baterias de Lítio?
\item  Como validar a eficácia dos preditores baseados em lógica fuzzy propostos na modelagem de temperatura em sistemas de gerenciamento de baterias de Lítio?
\end{itemize}
